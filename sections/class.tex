\section{Classification des surfaces}

\begin{defi}
    Soit $f:S\to \R$ une fonction de Morse sur une surface $S$. 
    Alors $f$ est "une bonne fonction de Morse" si $f$ est ordonnée et 
    si elle ne possède qu'un seul point critique d'indice $0$ et un seul 
    point critique d'indice $1$.
\end{defi}

\begin{prop}
    Soit $f:S\to\R$ une bonne fonction de Morse. Alors le nombre de points 
    critiques d'indice $1$ de $f$ est pair.
\end{prop}

\begin{proof}
    En effet, on sait que le passage d'une valeur critique d'indice $0$ rajoute
    une composante de bord, que le passage d'une valeur critique d'indice $2$ diminue 
    de un le nombre de composantes de bord. 
    Puis le passage d'une valeur critique d'indice $1$ rajoute ou diminue de un le nombre 
    de composantes de bord, puisque, $S$ étant orientable, on ne s'autorise pas la présence 
    de valeurs critiques d'indice $1$ dont le passage laisse le nombre de composantes de bord 
    inchangé. 
    Soit alors $n_0$ le nombres de valeurs critiques dont le passage ajoute une composante de 
    bord, $n_1$ le nombre qui fait diminuer le nombre de composantes de bord. 
    Après le passage de tous les points critiques, le nombre de composantes de bord est 
    $1+n_0-n_1-1=0$, car $S$ est sans bord, donc $n_0=n_1$.
\end{proof}

\begin{thm}
    Toute surface admet une bonne fonction de Morse.
\end{thm}

\begin{thm}[de classification des surfaces]
    Soit $S$ une surface, alors elle homéomorphe à $T_g$ avec $g\geq 0$.
\end{thm}

\begin{proof}
    Soit $f:S\to\R$ une bonne fonction de Morse. D'après une proposition précédente, $f$ admet 
    $2m$ points critiques associés à $2m$ valeurs critiques d'indice $1$. 
    Nous nommerons valeur critique d'indice $1$ du premier type toute valeur critique d'indice 
    dont le passage change le nombre de composantes de bord, et valeur critique d'indice $1$ 
    du second type celles dont le passage diminue le nombre de composantes de bord. 
    Quitte à modifier la fonction de Morse, on peut supposer les valeurs critiques d'indice 
    $1$ viennent par paires premier type second type. 

    Par récurrence, on montre que $S$ est homéomorphe à $T_m$. 
    Si $m=0$, c'est le théorème de Reeb.
    Sinon, lors du passage de la première valeur critique d'indice $1$, l'espace obtenu est un 
    disque auquel on a recollé deux côtés d'un rectangle le long de deux segments du bord, 
    espace homéomorphe à un cylindre ; lors du passage de la deuxième valeur critique, on 
    recolle les deux côtés d'un rectangle le long de deux segments, un par composante de 
    bord (il y en a deux).

    Soit $a$ une valeur critique intermédiaire située entre la deuxième valeur critique 
    d'indice $1$ et la troisième, alors on voit que $M_a$ est homéomorphe au tore privé 
    d'un disque, d'où $S=\mathbb T\#M^a$ où $M^a$ est une surface privée d'un disque 
    sur laquelle on peut définir une bonne fonction de Morse ayant $2m-2$ points critiques 
    d'indice $1$. Ce raisonnement permet d'enclencher une récurrence, ce qui achève de démontrer 
    le théorème.
\end{proof}

\begin{center}
    \begin{tikzpicture}[scale=1.2]
    \coordinate (p1) at (2.05,2.55);
    \coordinate (p2) at (3,2);
    \coordinate (p3) at (3.95,2.55);
    \path[draw, thick] plot [smooth cycle,tension=1] coordinates {(p1) (p2) (p3)};
    \draw[thick] (2,2.5) arc (180:360:1 and .16);
    \draw[very thick, arrow] (4, 2.25) -- (5,2.25);
    \coordinate (p1) at (5.05,2.55);
    \coordinate (p2) at (6,2);
    \coordinate (p3) at (6.95,2.55);
    \path[draw, thick] plot [smooth cycle,tension=1] coordinates {(p1) (p2) (p3)};
    \draw[thick] (5,2.5) arc (180:360:1 and .16);
    \draw[thick] (5.7,2.66) -- ++(0.2,-.32);
    \draw[thick] (6.1,2.66) -- ++(0.2,-.32);
    \draw[very thick, arrow] (7, 2.25) -- (8,2.25);
    \draw[thick] (8.05,2.55) -- ++(1.5,0);
    \draw[thick] (8.05,2) -- ++(1.5,0);
    \draw[thick] (8.05,2) arc (270:90:.1 and .275);
    \draw[thick,dashed] (8.05,2) arc (-90:90:.1 and .275);
    \draw[thick] (9.55,2) arc (270:90:.1 and .275);
    \draw[thick,dashed] (9.55,2) arc (-90:90:.1 and .275);
\end{tikzpicture}\\
    \textsc{Figure 5} – \textit{Espace obtenu après la première valeur critique (à gauche) et après la seconde (à droite)}
\end{center}
