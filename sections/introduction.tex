\section{Introduction}
La théorie de Morse, du mathématicien américain 
Marston Morse (1892 - 1972), part de l'idée que 
l'étude de fonctions très bien choisies sur une 
variété permet d'en étudier la topologie. 
C'est un ensemble de résultats de topologie et 
de géométrie différentielles qui, de manière 
plus précise, permet de relier l'étude de la 
topologie de l'espace sur lequel on définit 
certaines fonctions différentiables à l'étude 
des points critiques de ces fonctions différentiables.


Le thème de mon stage a été la théorie de Morse. 
Celui-ci a commencé le 27 mai 2024 et a duré 6 semaines.
Il a pris place au sein du laboratoire de mathématiques 
d'Orsay,  dans l'équipe de topologie et dynamique.
L'objectif était de me familiariser avec des notions et 
des éléments de la théorie de Morse pour pouvoir les 
appliquer dans la démonstration de la classification des 
surfaces compactes, connexes, orientables et sans bord.

Ce travail a commencé par l'étude de notions de base de 
topologie différentielle, notamment en étudiant le cours 
du même nom de Patrick Massot, ce qui a permis ensuite 
l'étude de différents livres portant essentiellement sur 
la théorie de Morse, tels que le célèbre \textit{Morse Theory} 
de J. Milnor ou le livre \textit{Théorie de Morse et Homologie de Floer} 
de Michèle Audin et Mihai Damian. Le présent document 
constitue le rapport du stage en question, où je résume 
ce que j'ai pu apprendre et en développant une preuve de 
la classification des surfaces compactes, connexes, orientables 
et sans bord.


Le rapport est découpé en trois parties.  Dans une première 
partie, que je nomme "Préliminaires", je présente quelques 
notions de base de topologie différentielle essentielles 
dans le travail qui suivra, où je montrerai notamment quelques 
résultats, comme le fait que toute variété compacte peut être 
plongée dans un espace affine, ou qu'un champ de vecteurs lisse 
qui s'annule en dehors d'un compact engendre un unique groupe 
à un paramètre de difféomorphes (un résultat qui sera utilisé 
dans la plupart des théorèmes de la section sur les fonctions 
de Morse sur les surfaces). La deuxième partie contient l'essentiel 
des résultats que j'ai vus pendant ce stage, commencera bien sûr 
par des définitions : qu'est-ce que c'est qu'une fonction de Morse ? 
Elle contiendra une preuve du lemme de Morse, une justification de 
l'existence de telles fonctions sur une variété et trois gros théorèmes 
qui concerneront les valeurs régulières d'une fonction de Morse sur une 
surface ainsi que le changement de topologie lors du passage d'une valeur 
critique.  Cette partie contiendra aussi quelques résultats concernant
les modifications de fonctions de Morse, essentielles pour la preuve de 
la classification. La dernière partie se sert enfin des résultats prouvés 
dans les sections précédentes pour établir la classification.


Je remercie évidemment ma tutrice de stage pour m'avoir accueilli, et 
de m'avoir invité à assister à certaines conférences à l'Institut Henri 
Poincaré notamment.

